\documentclass[aspectratio=169]{beamer}

\usepackage{makecell}
\usepackage{booktabs}

\usetheme{Boadilla}
\setbeamertemplate{navigation symbols}{}

% \title{Epidemic Broadcast}
% \subtitle{Performance Evaluation of Computer Systems and Networks}
% \author[Abdulwahed, Cima, Scatena]{Ahmed Khalil A. Abdulwahed, Lorenzo Cima, Niccolò Scatena}
% \institute[UNIPI]{University of Pisa}
% \date{2020/2021}
%
\begin{document}
%
% \begin{frame}
% 	\titlepage{}
% \end{frame}
%
% \section{Simulation model}
%
% \begin{frame}{The model}
% 	\begin{columns}
% 		\column{0.6\textwidth}
% 		\begin{itemize}
% 			\item 2D floorplan with users randomly dropped in
% 			\item A random user sends out a message; the other users
% 				relay it with a broadcast radius \emph{R}
% 			\item Time is \emph{slotted}
% 			\item \textbf{Trickle relaying}: when a user receives a
% 				message, it waits for \(T\) slots: message
% 				relayed if less than \(m\) copies received
% 		\end{itemize}
% 		\column{0.35\textwidth}
% 		\includegraphics[width=\textwidth]{img/snapshot}
% 	\end{columns}
% 	Modules:
% 	\begin{description}
% 		\item[Floorplan] the network representing the 2D
% 			floorplan
% 		\item[User] submodule representing a user
% 		\item[Oracle] used to terminate simulation and collect
% 			some stats
% 	\end{description}
% \end{frame}
%
% \begin{frame}{User module}
% 	\begin{columns}
% 		\column{0.5\textwidth}
% 		Finite State Machine
% 		\begin{description}
% 			\item[IDLE] state when started; slotting through
% 				self-messages
% 			\item[RECEIVING] hearing message in current slot
% 			\item[COLLISION] received another message in the same
% 				slot
% 			\item[HEARING] waiting for time window \(T\)
% 			\item[RELAYING] relay message if trickle relaying not
% 				triggered
% 		\end{description}
% 		\column{0.5\textwidth}
% 		\includegraphics[width=\textwidth]{img/userfsm}
% 	\end{columns}
% 	User messages are sent with a duration equal to
% 	\(\frac{slotDuration}{2}\) in order to ensure that it is
% 	received in current slot
% \end{frame}
%
% \begin{frame}{Model verification}
% 	\begin{columns}
% 		\column{0.5\textwidth}
% 		\begin{description}
% 			\item[Valgrind] no memory leaks
% 			\item[Graphical] execution inside QTEnv
% 			\item[Step-by-Step Debug] check that the correct code
% 				execution path is taken
% 			\item[Event Trace] check for correct scheduling of
% 				events
% 			\item[Deterministic] same inputs \textrightarrow{} same
% 				outputs
% 			\item[Degeneracy] config with low params values
% 			\item[Continuity] little input change \textrightarrow{}
% 				little output change (figure)
% 		\end{description}
% 		\column{0.5\textwidth}
% 		\includegraphics[width=\textwidth]{img/continuity-collisions}
% 	\end{columns}
% \end{frame}
%
% \section{Experiments design}
%
% \begin{frame}{Factors and Indexes}
% 	\begin{columns}
% 		\column{0.5\textwidth}
% 		Performance indexes:
% 		\begin{itemize}
% 			\item \textbf{Broadcast time} needed to cover a certain
% 				percentile of the users
% 			\item Final percentage of \textbf{covered users}
% 			\item \textbf{Energy efficiency}: depends on \(R\) and
% 				the number of messages sent
% 				\[\mathit{Eff} \propto \frac{1}{R \cdot M}\]
% 			\item \textbf{Collisions}
% 		\end{itemize}
% 		\column{0.5\textwidth}
% 		Tunable factors:
% 		\begin{itemize}
% 			\item Broadcast radius (\(R\))
% 			\item Trickle relaying hear window (\(T\))
% 			\item Trickle relaying max copies (\(m\))
% 			\item Maximum relay delay (\(\max(\delta)\)): introduced
% 				to avoid an issue with trickle relaying
% 		\end{itemize}
% 		Not tunable factors:
% 		\begin{itemize}
% 			\item Floorplan area (\(A\)) and dimensions ratio
% 				(\(\frac{X}{Y}\))
% 			\item User density (\(\frac{N}{A}\))
% 			\item Position of first user sending the message
% 		\end{itemize}
% 	\end{columns}
% \end{frame}
%
% \begin{frame}{Scenarios}
% 	\begin{columns}
% 		\column{0.65\textwidth}
% 		\begin{description}
% 			\item[High Density] \(A = 22500m^2\) (\(150m \times
% 				150m\)), \(N = 1125\mathit{users}\) (\(0.05
% 					\mathit{users}/m^2\))
% 			\item[Low Density] \(A = 250000m^2\) (\(500m \times
% 				500m\)), \(N = 1250\mathit{users}\) (\(0.005
% 					\mathit{users}/m^2\))
% 			\item[Rectangular] \(A = 30000m^2\) (\(300m \times
% 				100m\)), \(N = 1500\mathit{users}\) (\(0.05
% 					\mathit{users}/m^2\))
% 		\end{description}
% 		\column{0.35\textwidth}
% 		\begin{itemize}
% 			\item \(T \in [5s, 10s]\)
% 			\item \(\max(\delta) \in [5s, 10s]\)
% 			\item \(m \in [2, 6]\)
% 			\item \(R \in [30m, 50m]\) (low density)\\
% 				\(R \in [10m, 20m]\) (others)
% 		\end{itemize}
% 	\end{columns}
% 	\begin{block}{Analysis workflow}
% 		\(2^{k}r\) to spot most important factors for each index. Then,
% 		in-depth factorial analysis with the most important factors to
% 		study the behaviour between extreme values.
% 	\end{block}
% 	\begin{center}
% 		Jupyter notebooks: analysis automation
% 	\end{center}
% \end{frame}
%
% \section{Data analysis}
%
% \begin{frame}{High density (\(2^{k}r\))}
%     \begin{columns}
% 		\column{0.5\textwidth}
% 		Coverage always nearly perfect (above \(99\%\))\\[10pt]
% 		\begin{tabular}{l | c | cl}
% 			KPI & Factor & Percentage \\
% 			\hline \hline
% 			Collisions & R & \(66.27\%\) & \(\uparrow\) \\
% 				   & m & \(15.76\%\) & \(\uparrow\) \\
% 			\hline
% 			Messages & m & \(85.98\%\) & \(\uparrow\) \\
% 			\hline
% 			Broadcast Time & R & \(71.25\%\) & \(\downarrow\) \\
% 			& T & \(19.30\%\) & \(\uparrow\) \\
% 			\hline
% 		\end{tabular}\\[10pt]
% 		Excluding coverage (superior limit), we get low unexplained
% 		variations; higher for broadcast time (position starting node)
%
% 		To verify the assumption of finite variance for residuals of the
% 		broadcast time, a logarithmic transformation is needed: \(y' =
% 		\ln(y)\)
% 		\column{0.5\textwidth}
% 		\includegraphics[width=\textwidth]{img/hd/messages-m-perfplot}
% 	\end{columns}
% \end{frame}
%
% \begin{frame}{High density (optimizations)}
% 	\begin{columns}
% 		\footnotesize
% 		\column{0.5\textwidth}
% 		\begin{itemize}
% 			\item Good coverage with \(R \ge 8m\)
% 			\item Broadcast time can be improved by increasing \(R\)
% 				(sacrifice energy efficiency) or by reducing
% 				\(T\)
% 		\end{itemize}
% 		\begin{center}
% 			\includegraphics[width=0.8\textwidth]{img/hd/broadcasttime-R-ffplot}
% 		\end{center}
% 		\column{0.5\textwidth}
% 		\begin{itemize}
% 			\item Low \(m\) and high \(\max(\delta)\) reduce the
% 				number of collisions without sacrificing
% 				anything else
% 			\item Good linear relationship between collisions and
% 				\(\max(\delta)\)
% 		\end{itemize}
% 		\begin{center}
% 			\includegraphics[width=0.75\textwidth]{img/hd/collisions-D-ffplot}
% 		\end{center}
% 	\end{columns}
% \end{frame}
%
% \begin{frame}{Low density (\(2^{k}r\))}
% 	\begin{columns}
% 		\column{0.5\textwidth}
% 		Also in this case the coverage always reach high values (\(> 99\%\))
%
% 		\begin{table}
% 			\begin{tabular}{l | c | c}
% 				KPI & Factor & Percentage \\
% 				\hline \hline
% 				Collisions & R & \(58.96\%\) \\
% 				& m & \(19.26\%\) \\
% 				\hline
% 				Messages & m & \(85.18\%\) \\
% 				\hline
% 				Broadcast Time & R & \(58.90\%\) \\
% 				& T & \(30.43\%\) \\
% 				\hline
% 			\end{tabular}
% 			\caption{Most influencing factors for parameters}
% 		\end{table}
% 		\column{0.5\textwidth}
% 		\begin{figure}
% 		    \includegraphics[height=0.65\textheight]{img/ld/broadcasttime-R-perfplot}
% 		    \caption{Increase R reduces so much the broadcast time (but increases the number of collisions and the energy consumed)}
% 		\end{figure}
% 	\end{columns}
% \end{frame}
%
% \begin{frame}{Low density (optimizations)}
% 	\footnotesize
% 	\begin{columns}
% 	    \column{0.5\textwidth}
% 	        \begin{itemize}
% 	        \item The minimum value of R to have a good coverage (\(> 99\%\)) is 25m
% 	        \item To improve energy efficiency, avoid to increase
% 			the broadcast radius (T can be increase
% 			instead). This also reduces the number of collisions
% 	        \end{itemize}
% 	        \begin{center}
% 	            \includegraphics[width=0.7\textwidth]{img/ld/messages-R-ffplot.png}
% 	        \end{center}
% 	    \column{0.5\textwidth}
% 	        \begin{itemize}
% 	        \item A low number of max copies implies great benefits for all parameters
% 	        \item Trade-off between broadcast time and energy efficiency for the hear window parameter
% 	        \end{itemize}
% 	        \begin{center}
% 	            \includegraphics[width=0.7\textwidth]{img/ld/broadcasttime-T-ffplot.png}
% 	        \end{center}
% 	\end{columns}
% \end{frame}
%
% \begin{frame}{Rectangular  (\(2^{k}r\))}
% 		\begin{columns}
% 		\column{0.5\textwidth}
% 		 \begin{itemize}
% 		     \item
% 	Coverage almost always reaches 100\%.The lowest value is 99.6664\%. \\[5pt]
%
% 		\begin{table}
% 			\begin{tabular}{l | c | c}
% 				KPI & Factor & Percentage \\
% 				\hline \hline
% 				Collisions & R & \(64.31 \%\) \\
% 				& m & \(16.47\%\) \\& Rm& \(9.59 \%\) \\
% 				\hline
% 				Messages & m & \(84.89\%\) \\& R & \(5.58\%\) \\
% 				\hline
% 				Broadcast Time & R & \(65.40\%\) \\
% 				& T & \(20.58\%\) \\
% 				\hline
% 			\end{tabular}
% 			\caption{Most influencing factors for parameters}
% 		\end{table}
% 		\item the shape of the plane has a negligible influence, so the obtained results for the optimizations are similar to previous ones
% 		\end{itemize}
% 		\column{0.5\textwidth}
% 				\begin{figure}
% 		    \includegraphics[height=0.65\textheight]{img/rect/collisions_m_perfplot.png}
% 		    \caption{Decrease the maximum number of copies to decrease the total
% number of collisions}
% 		\end{figure}
%
% 	\end{columns}
% \end{frame}
%
% \begin{frame}{Position of starting node}
% 	\begin{columns}
% 		\column{0.5\textwidth}
% 		The broadcast time changes with the initial position of the
% 		first user. In the corner, the mean and the minimum broadcast
% 		time are higher than in the center and in the border.\\[10pt]
% 		\begin{tabular}{lcccc}
% 		\multicolumn{5}{c}{Low density (98th percentile broadcast time)}\\
% 		\toprule
% 		Start Node Pos\@. & Mean & Std\@. Dev\@. & Min\@. & Max\@. \\
% 		\midrule
% 		Center & \(35.166667s\) & \(1.533158s\) & \(31s\) & \(40s\) \\
% 		Border & \(53.7s\) & \(1.914554s\) & \(50s\) & \(58s\) \\
% 		Corner & \(65.1s\) & \(2.186952s\) & \(61s\) & \(70s\) \\
% 		\bottomrule
% 	    \end{tabular}
% 		\column{0.5\textwidth}
% 		    \begin{center}
% 		    \includegraphics[height=0.7\textheight]{img/ld/start-node-coverage.png}
% 		    \end{center}
% 	\end{columns}
% 	The coverage drops down for low values of \(R\) when the position of the
% 	starting node is in the corner/border. In this points, we also have
% 	higher variance due to catastrophic situations where the network is not
% 	able to reach anyone
% \end{frame}
%
% \section{Conclusions}
%
% \begin{frame}{Conclusions}
% 	\large
% 	\begin{itemize}
% 		\item Trade-off needed between broadcast time and energy
% 			efficiency on the \(R\) and \(T\) parameters
% 		\item Coverage almost always perfect with a good \(R\), except
% 			in some catastrophic cases when the starting node is in
% 			the corner. The probability to reach at least a user can
% 			be computed as:
% 			\[
% 				P = 1 - {\left(\frac{XY - \alpha\pi R^2}{XY}\right)}^{N-1}
% 			\]
% 		\item Trickle relaying is very effective to reduce collisions
% 			when using low \(m\)
% 		\item Shape of the floorplan generally irrelevant: slight
% 			increase in the broadcast time
% 		\item Position of the starting node, possibly in the center, is
% 			very important to have a good coverage and a low
% 			broadcast time
% 	\end{itemize}
% \end{frame}
%
% \begin{frame}
%     \centering
%     \Huge \color{blue} Thank you!
% \end{frame}
%
\end{document}
