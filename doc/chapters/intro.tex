\chapter{Introduction}\label{ch:intro}

\ldots TODO \ldots

\chref{ch:background} provides a description of the Bitcoin system, an
explanation of the \omnetpp{} simulation framework, and a review of the
state-of-the-art in blockchain simulation presenting some available simulators.

\chref{ch:implementation} outlines the design and implementation of \iblock{}.

\chref{ch:testing} presents the test conducted to verify and validate the
simulator.

\chref{ch:performance} illustrates the optimizations done on \iblock{} and an
evaluation of his performance in terms of speed and memory usage, comparing it
with another simulator.

\chref{ch:pocs} showcases some proofs of concept implemented using \iblock{}
and some examples of analyses that can be done using the simulator.

\chref{ch:next} discusses some limitations of the current implementation of
\iblock{} and possible future work to improve it.

\chref{ch:conclusion} concludes the thesis with some final remarks.
\appendixref{appendix:computations} contains the mathematical computations for
the time needed to mine a new Bitcoin block and the network's difficulty
adjustment.
