\chapter{Introduction}\label{ch:intro}

Bitcoin, introduced by the pseudonymous Satoshi Nakamoto in 2008
\cite{whitepaper}, is the first decentralized cryptocurrency. Its launch marked
a revolutionary step in digital finance, enabling peer-to-peer transactions
without the need for a trusted central authority. Over the years, Bitcoin and
other blockchain-based technologies have gained widespread attention,
influencing industries, governments, and academia alike.

Blockchain technology has since evolved beyond cryptocurrencies, finding
applications in supply chain management, voting systems, healthcare, and more.
With this growing interest, understanding the behavior and performance of
blockchain networks under various conditions has become crucial. Simulation
tools play a key role in this regard, allowing researchers and developers to
test blockchain networks in a controlled environment, optimizing their designs
and analyzing their performance.

\paragraph{Challenges in Blockchain Simulation}\label{par:intro-challenges}

Numerous blockchain simulators have been developed in recent years, each
attempting to address specific research or application needs. These tools are
typically used to model various aspects of blockchain systems, such as
transaction propagation, block creation, and consensus mechanisms, under
diverse scenarios. However, as discussed in further detail in
\secref{sec:simulators}, existing simulators face significant limitations:
\begin{itemize}
	\item \textbf{Incomplete Modeling}: No single simulator accurately
		models all facets of a real-world Bitcoin (or other blockchain)
		network. Key features like realistic network topology,
		UTXO-based transaction accounting, or variable transaction fees
		are often absent or oversimplified;
	\item \textbf{Performance Bottlenecks}: Simulating large-scale networks
		or long time periods requires high computational efficiency.
		Most simulators, however, are built using high-level
		interpreted languages (e.g., Python, Java, Go), which
		inherently limit performance;
	\item \textbf{Lack of Extensibility}: Many existing simulators lack
		modular designs or are not based on established simulation
		frameworks, making it challenging to extend them or integrate
		new features;
	\item \textbf{Limited Security Analysis}: Few simulators allow the
		modeling of malicious behavior, such as selfish mining or other
		attacks, restricting their use for studying blockchain
		security.
\end{itemize}

\paragraph{The Proposed Solution}\label{par:intro-solution}

This thesis proposes \iblock{}: a novel, extensible, and detailed Bitcoin
network simulator. \iblock{} is designed to address the limitations of existing
simulators and provide a comprehensive tool for blockchain simulation. It is a
discrete-event simulation library implemented in C++ and built on top of
\omnetpp{}, a widely used framework for simulating networks and systems. By
leveraging \omnetpp{}'s capabilities, \iblock{} combines high performance with
modularity, scalability, and extensibility.

Key features of \iblock{} include:
\begin{itemize}
	\item \textbf{Detailed Protocol Modeling}: \iblock{} implements a
		realistic Bitcoin protocol, including components such as
		blocks, transactions, mempools, and peers. The UTXO-based
		transaction accounting model is fully supported, offering a
		more accurate representation of Bitcoin's data layer. Although
		the protocol is not fully implemented in its entirety, block
		and transaction creation and propagation are modeled in detail
		and \iblock{} provides a solid foundation for further
		development aimed towards a complete Bitcoin protocol model;
	\item \textbf{Scalability}: Designed to handle large-scale networks,
		\iblock{} leverages C++'s performance advantages to simulate
		longer timeframes and more extensive node networks compared to
		existing simulators;
	\item \textbf{Extensibility}: Built with modularity in mind, \iblock{}
		allows researchers to easily add new features or modify
		existing ones. For example, alternative consensus mechanisms or
		nodes with malicious behaviors can be implemented with minimal
		effort;
	\item \textbf{Integration with \omnetpp{} Libraries}: By being part of
		the \omnetpp{} ecosystem, \iblock{} may be in future integrated
		with other libraries, such as INET, to simulate network-layer
		effects like bandwidth constraints, latencies, and link
		failures.
\end{itemize}

The primary goal of this thesis is to design and implement \iblock{} as a
versatile tool for blockchain research. The work focuses on developing a
high-fidelity simulation library that accurately models Bitcoin's core
functionality, while providing good performance, modularity and extensibility.

There is still some work that needs to be done before considering \iblock{} a
complete simulator, such as adding support for \omnetpp{}'s channel and the
Bitcoin's gossip protocol, but the current implementation provides a solid
foundation for a generic Bitcoin simulator that does not watch only at specific
aspects of the protocol, such as many existing simulators do, and tries to keep
an eye on every aspect in detail.

\section{Organization of this Work}\label{sec:organization}

\chref{ch:background} describes the Bitcoin system, illustrates the \omnetpp{}
simulation framework, and reviews the state-of-the-art in blockchain simulation
discussing some available simulators.

\chref{ch:implementation} details the design and implementation of \iblock{}.

\chref{ch:testing} outlines the test conducted to verify and validate the
simulator.

\chref{ch:performance} examines optimizations applied to \iblock{} and
evaluates its performance in terms of speed and memory usage, comparing it
with another simulator.

\chref{ch:pocs} showcases some proofs of concept implemented with \iblock{}
and provides some examples of analyses achievable with the simulator.

\chref{ch:next} discusses the limitations of the current \iblock{}
implementation and proposes areas for future enhancement.

\chref{ch:conclusions} concludes the thesis with final remarks, while
\appendixref{appendix:math} contains mathematical computations for estimating
the time required to mine a new Bitcoin block and the formulas for the
network's difficulty adjustments.

