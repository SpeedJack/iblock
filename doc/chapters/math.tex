\chapter{Mathematical Background}\label{appendix:math}

This appendix shows the mathematical foundations used to compute the time
needed to mine a new block and to adjust the network's mining difficulty. All
the mathematics presented here is based on informations extracted from the
\citetitle{bitcoin-dev} \cite{bitcoin-dev} and the
\citetitle{btcwiki-difficulty} \cite{btcwiki-difficulty}.

\section{Mining Difficulty and Target}\label{appendix:difficulty-target}

The current network difficulty is encoded in the \code{nBits} field of the last
block's header. The field is a 4-byte integer that represent the maximum
allowed hash for a block to be considered valid, and is usually called
``target''. The hash is encoded in compact form, explained in
\secref{subsec:impl-utility}.

Difficulty is always computed as relative to the minimum difficulty of 1, for
which corresponds the maximum target:
\begin{verbatim}
0x00000000ffff0000000000000000000000000000000000000000000000000000
\end{verbatim}
or \(\texttt{0x1d00ffff}_{c}\) in compact form (the subscript \code{c} denotes
the compact form).

So the current difficulty \(D\) can be computed from the current target \(T\)
and the maximum target \(T_{\text{max}}\) using \eqref{eq:target2difficulty}:

\begin{equation}\label{eq:target2difficulty}
	D = \frac{T_{\text{max}}}{T}
\end{equation}

A faster way to compute the difficulty is to use directly the compact form of
the target, by operating on the mantissa and exponent. Call \(m\) the mantissa
of the target \(T\), and \(e\) its exponent. Call \(M\) the mantissa of the
maximum target \(T_{\text{max}}\), and \(E\) its exponent. The difficulty can
be computed as in \eqref{eq:nbits2difficulty}.

\begin{equation}\label{eq:nbits2difficulty}
	D = \frac{M}{m} \times 2^{2^{3}(E - e)} = \frac{M}{m} \times 256^{(E -
	e)}
\end{equation}

The target can be computed from the difficulty using
\eqref{eq:difficulty2target}, where the formula expansion for
\(T_{\text{max}}\) has been derived from \eqref{eq:compact2hash} from page
\pageref{eq:compact2hash} substituting \(M = \texttt{0xffff} = 2^{16} - 1\) and
\(E = \texttt{0x1d} = 29\).

\begin{equation}\label{eq:difficulty2target}
	T = \frac{T_{\text{max}}}{D} = \frac{(2^{16} - 1) \times 256^{(29 -
	3)}}{D}
\end{equation}

\section{Time to Mine a Block}\label{appendix:time-to-mine}

The \code{Miner} application needs to schedule a timer to simulate the time
needed to mine a block. So, it must know in advance how long it will take to
mine the next block, based only on the current network difficulty and its hash
rate.

Given the hash rate of the miner \(h\) and the network difficulty \(D\), the
average time to mine a block is given by \eqref{eq:time-to-mine}.

\begin{equation}\label{eq:time-to-mine}
	\overline{t} = \frac{D \times 2^{32}}{h}
\end{equation}

Method \code{getTimeToBlock()} in the \code{Miner} class computes the
difficulty as in \eqref{eq:nbits2difficulty} and then computes mean time to
mine the next block as in \eqref{eq:time-to-mine}. The computed value is then
used as a mean for an exponential distribution\footnote{Mining is a process in
which the miner insert different \emph{nonces} inside the block's header until
it finds a hash that is lower than the target. The distribution of the time is
then memory-less, as probability of success does not change after each trial,
and the exponential distribution is the most suitable.} from which the time to
mine the next block is drawn.

\section{Initial Network Difficulty}\label{appendix:initial-difficulty}

Before starting the simulation, the \code{GBM} module computes the initial
difficulty from which derives the initial target and puts it in the genesis
block's header in the \code{nBits} field. The work is done by the
\code{computeInitialNBits()} method.

First, the \code{GBM} gets the hash rate from all the miners in the network.
Call the total hash rate of the network \(H\). We need to compute the
difficulty \(D\) such that the network with \(H\) hash rate finds a block in
\(600\) seconds (10 minutes). From \(D\), we then need to derive the target,
which is the maximum allowed hash for a block to be considered valid.

The time needed by the entire network to mine a block at minimum difficulty
(\(D_{\text{min}} = 1\)) is given by \eqref{eq:time-to-mine} substituting the
hash rate of the network \(H\) for the miner's hash rate \(h\) and \(D =
D_{\text{min}} = 1\). The needed initial network difficulty \(D\) can be
computed from the ratio \(R\) between the time needed to mine a block with the
network hash rate and the expected time to mine a block (\(600\) seconds) as in
\eqref{eq:next-difficulty}.

\begin{equation}\label{eq:next-difficulty}
	D = \frac{D_{\text{min}}}{R} = \frac{D_{\text{min}}}{t/600} =
	\frac{1}{2^{32}/600H} = \frac{600 H}{2^{32}}
\end{equation}

From the difficulty \(D\) and the maximum target \(T_{\text{max}} =
\texttt{0x1d00ffff}_{c}\), we can compute the needed target as in
\eqref{eq:difficulty2target}. The target is computed as a big double number,
then the \code{Hash::fromBigDouble()} method is used to convert the number to
an instance of the \code{Hash} class which approximates the result in order to
store it in compact form for the \code{nBits} field.

\section{Network Difficulty Adjustment}\label{appendix:difficulty-adjustment}

Every \(2016\) blocks, each \code{BlockchainManager} in the network
indipendently adjust the network difficulty. The newly computed target is then
added to the \code{nBits} field of the next block's header. Method
\code{getNextTargetNBits()} does the job.

First, the \code{BlockchainManager} gets the previous difficulty
\(D_{\text{prev}}\) from the target store in the current block's header using
\eqref{eq:nbits2difficulty}. Then, it gets the time difference between the last
block and the block \(2015\)\footnote{\emph{Sic!} There is an \emph{off-by-one
error} (also ``Time Warp Bug'')
in the \citetitle{bitcoin-core} implementation that causes the difficulty to be
calculated using timestamps from only \(2015\) blocks while still using the
expected time for \(2016\) blocks \cite{bitcoin-core}. To avoid forks, the bug
has never been fixed --- so \iblock{} does an off-by-one calculation as well!}
blocks before --- call this difference \(\Delta T\). The expected time to mine
\(2016\) blocks is \(T_{\text{expected}} = 2016 \times 600 = 1209600\) seconds.
The ratio between the actual time and the expected time is \(R = \frac{\Delta
T}{T_{\text{expected}}}\).

As per the Bitcoin protocol specification, the ratio is then bounded between
\(0.25\) and \(4\). The new difficulty is then computed as \(D =
D_{\text{prev}} \times R\), where \(D_{\text{prev}}\) is the previous
difficulty.

The new difficulty is then converted to the new target using
\eqref{eq:difficulty2target}.
