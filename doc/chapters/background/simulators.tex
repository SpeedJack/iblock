\section{Blockchain Simulators}\label{sec:simulators}

A wide range of blockchain simulators is available today, with most being free
and open-source. These simulators aim to model different aspects of blockchain
systems, including the network, consensus, and data layers. This section
provides an overview of the most widely used blockchain simulators, their key
features, and their limitations.

\subsection{State of the Art}\label{subsec:state-of-the-art}

The current landscape of blockchain simulators has been extensively reviewed in
\cite{simureview}, which compares the features and capabilities of various
tools. Among the most notable simulators discussed are:
\begin{description}
	\item[Bitcoin-Simulator] Developed in C++ in
		\citeyear{bitcoin-simulator} for the ns-3 network simulator
		\cite{ns3}. It supports realistic and geographically
		distributed networks. It does not model transaction fee
		distribution and dynamic difficulty adjustments
		\cite{bitcoin-simulator};
	\item[BlockSim (Faria's Version)] Developed in Python in
		\citeyear{blocksim-faria} for the SimPy simulation framework
		\cite{simpy}. One of the most complete simulators but does not
		supports variable sized blocks and transactions, block interval
		distribution, transaction fee distribution and dynamic
		difficulty adjustments \cite{blocksim-faria};
	\item[SimBlock] Developed in Java in \citeyear{simblock}, it lacks
		support for block size distribution and transactions are not
		supported altogether \cite{simblock};
	\item[BlockSim (Alharby's Version)] Developed in Python in
		\citeyear{blocksim} it does not model the network layer and
		does not support dynamic difficulty adjustments and variable
		block sizes \cite{blocksim};
	\item[VIBES] Developed in Scala in \citeyear{vibes} does not model the
		network altogether and block and transactions are supported
		only at a basic level \cite{vibes}.
\end{description}

There is no simulator that supports the UTXO-based transaction accounting
model.

\subsubsection{Key Insights}\label{subsubsec:key-insights}

\begin{description}
	\item[Network Layer] Bitcoin-Simulator, BlockSim (Faria), and SimBlock
		consider the geographical distribution of nodes, enabling more
		realistic simulations of global networks. However, VIBES and
		BlockSim (Alharby) lack this capability;
	\item[Consensus Layer] Only SimBlock supports alternative consensus
		mechanisms, such as Proof-of-Stake. However, some simulators,
		such as BlockSim (Alharby) and SimBlock, assume all nodes act
		honestly, making them unsuitable for analyzing scenarios
		involving malicious actors;
	\item[Data Layer] Several simulators --- BlockSim (Faria), VIBES,
		BlockSim (Alharby) --- simulate transaction propagation across
		the network. However, none implement the UTXO-based transaction
		account modeling. BlockSim (Alharby) is the only simulator
		supporting variable transaction fees.
\end{description}

\subsubsection{Limitations of Existing Simulators}\label{subsubsec:limitations}

Despite their capabilities, most blockchain simulators share the following
limitations:
\begin{itemize}
	\item \textbf{Lack of Realistic Network Modeling}: Few simulators
		include detailed simulations of network, links, bandwidths, and
		latencies; Those that do, have other limitations in the
		consensus and data layers;
	\item \textbf{No Support for UTXO Models}: None of the simulators
		accurately model the UTXO structure and the UTXO-based
		accounting model;
	\item \textbf{Simplistic Data Representation}: Often, only essential
		fields of blocks and transactions are modeled, limiting
		realism and flexibility;
	\item \textbf{Poor Extensibility}: Many simulators lack modular design
		or are not based on established simulation frameworks, making
		feature additions difficult;
	\item \textbf{Limited Security Analysis}: Few simulators can model
		malicious nodes or test network vulnerabilities; Most
		simulators assume all nodes are honest;
	\item \textbf{Performance Bottlenecks}: Many are slow and
		memory-intensive, restricting scalability to larger networks or
		longer simulations.
\end{itemize}

\iblock{} was developed as a direct response to these limitations, aiming to
deliver a high-performance, extensible blockchain simulation framework.

\subsection{BlockSim}\label{subsec:blocksim}

BlockSim, developed by \citeauthor{blocksim} \cite{blocksim}, is widely
regarded as one of the most comprehensive blockchain simulators. Its primary
strength lies in its robust implementation of the consensus layer. However,
BlockSim has several critical drawbacks that limit its usability and
efficiency.

\subsubsection{Features and Capabilities}\label{subsubsec:blocksim-features}

\begin{itemize}
	\item \textbf{Modes of Operation}: BlockSim supports two modes:
		\begin{description}
			\item[Light Mode] Transactions are simplified and not
				propagated across the network; Transactions are
				created during simulation when a new block is
				mined and directly added to it. This mode
				allows for fast speed and low memory usage;
			\item[Full Mode] Transactions are propagated across the
				network and modeled with more details (altough
				no multiple-input multiple-output (MIMO)
				transactions are supported and UTXO-based
				accounting is not considered), providing a
				higher level of detail but at the cost of
				performance and memory consumption. In this
				mode, transactions are all created at the start
				of the simulation, causing the software to go
				out of memory in big setups even before
				starting the simulation.
		\end{description}
	\item \textbf{Consensus Layer}: BlockSim includes an extensive
		implementation of consensum mechanism, though it does not model
		UTXOs and MIMO transactions and it assumes honest behavior by
		all nodes, making it unsuitable for security-focused
		simulations;
\end{itemize}

\subsubsection{Limitations}\label{subsubsec:blocksim-limitations}

\begin{description}
	\item[Performance] In ``Full' mode, BlockSim is exceptionally slow and
		resource-intensive. Simulating even small networks of a few
		dozen nodes for several hours of simulated time can take days
		and require significant memory;
	\item[Simplified Data Layer] BlockSim does not implement a UTXO-based
		accounting model: nodes can generate transactions without
		verifying their available funds, reducing accuracy. Moreover,
		transactions are limited to a single input and a single output;
	\item[Limited Features] BlockSim lacks dynamic difficulty adjustments
		and does not account for malicious behavior;
	\item[Limited Statistics] The range of statistics collected by BlockSim
		is narrow, limiting the depth of analysis possible. Most
		statistics are also meaningless: for example, BlockSim creates
		blocks of fixed size and collects the size of every block,
		which is just a constant;
	\item[Extensibility] BlockSim is developed in Python but does not
		utilize a simulation framework like \omnetpp{}. This makes
		extending the simulator challenging and time-consuming;
	\item[Scalability] The simulator struggles to handle larger networks or
		longer durations due to its inefficiencies in memory and
		computational resource usage.
\end{description}

In \secref{sec:comparison}, \iblock{} will be compared with BlockSim in terms
of performance.
