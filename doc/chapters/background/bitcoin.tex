\section{Bitcoin}\label{sec:bitcoin}

This section offers an overview of Bitcoin, the digital currency
introduced by \citeauthor{whitepaper} in \citeyear{whitepaper} with his famous
white paper \cite{whitepaper}. The focus here is on aspects of the Bitcoin
protocol that are specifically relevant to this thesis project. Readers who are
already well-acquainted with Bitcoin may choose to skip this section.

The following sources provide the foundation for the information presented and
may serve as additional references for readers interested in further details:
\begin{itemize}
	\item The original Bitcoin white paper \cite{whitepaper};
	\item The \citetitle{bitcoin-wiki} \cite{bitcoin-wiki};
	\item The \citetitle{bitcoin-dev} \cite{bitcoin-dev};
	\item The website maintained by \citeauthor{learnmeabitcoin},
		\citetitle{learnmeabitcoin} \cite{learnmeabitcoin};
	\item The reference implementation of Bitcoin, \citetitle{bitcoin-core}
		\cite{bitcoin-core}.
\end{itemize}

\subsection{The Blockchain}\label{subsec:blockchain}

Bitcoin is a decentralized digital currency that operates on a peer-to-peer
network of nodes. The foundation of Bitcoin is a public ledger called the
blockchain, which records all transactions that have ever occurred on the
network.

The blockchain is a chain of blocks, where each block contains some metadata
(the block header) and a list of transactions. Blocks are chained in
chronological order, with each block header containing an hash to the previous
block header. The header additionally contains the hash of all transactions in
the block\footnote{Actually, it contains the root of a Merkle tree of all
ordered transactions.}, thus any change to a transaction would require a change
in the block header, which would in turn require a change in all subsequent
block headers. This makes the blockchain tamper-evident and highly
tamper-resilient, as computing block header hashes --- a job called ``mining''
--- is computationally expensive.

The block header fields are listed in Table~\ref{tab:block-header}. A block
header is \(80\) bytes long, and the total block size, including transactions,
is limited to \(1\) MB.

\begin{table}[tbhp]
	\centering
	\begin{tabularx}{\linewidth}{|l|X|}
		\toprule
		\textbf{Field} & \textbf{Description} \\
		\midrule
		Version & The version of the block format. \\
		\midrule
		Previous block hash & The hash of the previous block header. \\
		\midrule
		Merkle root & The root of a Merkle tree of all transactions in
		the block. \\
		\midrule
		Time & The time the block was created. \\
		\midrule
		nBits & The target value for the block hash. \\
		\midrule
		Nonce & A value that miners change to try to find a valid block
		hash. \\
		\bottomrule
	\end{tabularx}
	\caption{Block header fields.}\label{tab:block-header}
\end{table}

The blockchain is secured by a process called mining, in which nodes compete to
solve a cryptographic puzzle. The puzzle consists of finding a block header
hash that is below a certain target value, demonstrating \emph{Proof-Of-Work}
(PoW). The target value is adjusted every \(2016\) blocks by the network and
stored in the \code{nBits} field (see
\appendixref{appendix:difficulty-adjustment}). Miners use the \code{Nonce}
field to change the block header hash, and the first miner to find a valid hash
is allowed to broadcast the block to the network and earn rewards. Since miners
need to compute hashes to solve the puzzle, their speed is measured in hashes
per second, called \emph{hash rate}.

Nodes in the network propagate blocks to their peers, following a gossip
protocol. When a node receives a block, it validates it by checking that the
block respects the \emph{consensus rules} before adding it to its local copy of
the blockchain.

Sometimes two miners find a valid block more or less at the same time, creating
a \emph{fork} in the blockchain\footnote{This may also happen due to slow data
links between nodes or temporary network partitions.}. When this happens, the
network follows the rule of the longest chain, which states that the chain with
the most accumulated work \idest{the one with the highest \emph{height},
meaning it has more blocks} is the valid chain (parities are broken in favor of
the first received chain). The network eventually reaches a consensus on which
chain to follow, and the other chain is not considered until it becomes the
longest chain. When this happens, the network performs a \emph{chain
reorganization} to switch to the longest chain.

Also changing the consensus rules can create a fork. When the Bitcoin software
is updated with new rules, the following situations may happen:
\begin{enumerate}
	\item Upgraded nodes reject blocks that do not respect the new rules,
		while non-upgraded nodes accept both the old and the new blocks
		(backward-compatible upgrade). This creates a fork in the
		network. If upgraded nodes control the majority of the
		network's hash rate, they will eventually create a longer
		chain, since non-upgraded nodes will also accept the blocks
		following the new rules, thus keeping the blockchain from
		permanently diverging. This is called a \emph{soft fork};
	\item Non-upgraded nodes reject blocks that respect the new rules. This
		creates a fork in the network with two divergent chains --- one
		for non-upgraded nodes and one for upgraded nodes --- that will
		remain until all nodes in the network are upgraded. This is
		called a \emph{hard fork} and usually produces a new currency.
\end{enumerate}

An example of a hard fork is the creation of \citetitle{bitcoin-cash} in
\citeyear{bitcoin-cash}, which occurred due to a disagreement in the Bitcoin
community about the block size and the scalability of the network
\cite{bitcoin-cash}.

\subsection{Transactions}\label{subsec:transactions}

Transactions are used to transfer bitcoins between \emph{wallets}. A wallet is
a software that generates and stores a pair of public and private keys. Coins
can be sent to a wallet by means of \emph{wallet addresses}, which are derived
from the public key. Wallets can have multiple addresses, and addresses can be
used multiple times (although it is not recommended for privacy reasons).

Transactions are composed of \emph{inputs} and \emph{outputs}. Inputs reference
previous outputs and spend them, while outputs specify the amount to be sent
and the new coins' owner. Until spent, the output of a transaction is called an
\emph{Unspent Transaction Output} (UTXO).

The difference between the sum of inputs and the sum of outputs is the
\emph{transaction fee}, which is collected by the miner who includes the
transaction in a block. When a miner successfully mines a block, it is allowed
to include a special transaction called the \emph{coinbase transaction}, which
has no inputs and generates new coins and collects the transaction fees from
all the other transactions in the block. The total value of the outputs in a
coinbase transaction is called the \emph{block reward} and is composed by the
sum of the \emph{block subsidy} and the transaction fees. The block subsidy is
a fixed amount of coins that halves every \(210,000\) blocks, or roughly every
4 years, until it reaches zero --- these newly \emph{minted} coins are the only
way bitcoins are created.

Table~\ref{tab:tx-fields} lists the fields of a transaction. An input
references a previous output by including the transaction's hash and the output
index (together called ``outpoint''). The output index is used to select which
output to spend when a transaction has multiple outputs. An output specifies
the amount to be sent and a new owner, by the mean of a \emph{locking script}
called \emph{pubkey script}. The locking script can be unlocked by the new
owner with an input containing a \emph{signature script} that satisfies the
conditions specified in the locking script. Those scripts are programs coded in
a special language processed by a non-Turing-complete \emph{stack machine}.
Additional details about these scripts are not provided here since they are not
particularly relevant to this thesis project.

\begin{table}[tbhp]
	\centering
	\begin{tabularx}{\linewidth}{|l|X|}
		\toprule
		\textbf{Field} & \textbf{Description} \\
		\midrule
		Version & The version of the transaction format. \\
		\midrule
		Input count & The number of inputs in the transaction. \\
		\midrule
		Inputs & A list of inputs, each referencing a previous output.
		\\
		\midrule
		Output count & The number of outputs in the transaction. \\
		\midrule
		Outputs & A list of outputs, each specifying the amount to be
		sent and the new owner. \\
		\midrule
		Locktime & A field that specifies when the transaction can be
		included in a block. \\
		\bottomrule
	\end{tabularx}
	\caption{Transaction fields.}\label{tab:tx-fields}
\end{table}

Transactions are broadcast to the network, and nodes validate them before
adding them to their \emph{mempool}. The mempool is a pool of
\emph{unconfirmed} transactions. ``Unconfirmed'' means that the transaction has
not yet been included in a block. Miners select transactions from the mempool,
usually by choosing those with the highest transaction fees, and include them
in the blocks they mine. The number of confirmations of a transaction is the
number of blocks that have been mined on top of the block that contains the
transaction. If a chain reorganization happens, the transaction included in the
old chain are returned to the mempool, unless they are included in the new
chain as well. Thus, the number of confirmations is a measure of how improbable
it is that a transaction will be reversed.

If the same coins are spent in two different transactions
(\emph{double-spending}), the network will eventually reach a consensus on a
single transaction to include in the blockchain. The transaction that is not
included will remain in the mempool, as the network will reject any block that
includes a double-spending transaction. Since there is no pre-defined order of
transactions, users should wait for some confirmations before considering a
transaction settled.
