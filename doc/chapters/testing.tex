\chapter{Testing the Simulation Model}\label{ch:testing}

This chapter outlines the tests conducted to verify the simulation model.

Two types of tests should be perfomed on every newly developed simulation
model: verification and validation tests. Verification and validation are
distinct but complementary processes critical to building confidence in the
model's reliability and effectiveness. Verification confirms that the model
accurately implements the conceptual design, while validation assesses the
model's ability to represent real-world system behavior
\cite[Chapter~15]{omnetpp-simulation-manual}.

Verification is the process of ensuring that the simulation model has been
correctly built in accordance with the conceptual model. This phase is focused
on detecting and eliminating any discrepancies between the implemented model
and its initial design, concentrating on accuracy in logic, consistency in data
flows, and correctness of algorithms. In this project, verification tests were
designed to catch issues such as programming errors, incorrect parameter
settings, and unintended exceptions that might disrupt the simulation.

Verification thus serves as an internal consistency check, confirming that the
logic and processes coded in the model adhere strictly to the specifications of
the conceptual model.

On the other hand, validation is the process of determining whether the
simulation model accurately represents the actual system it is meant to
emulate. Unlike verification, which focuses on the internal consistency of the
model with its design, validation is an external assessment that compares the
model's outputs to real-world data or expected behavior patterns. In this
phase, tests are conducted to evaluate whether the model produces results that
are consistent with observed data or accepted theoretical outcomes of the real
system.

Validation tests are essential for ensuring that the simulation's predictions,
trends, and responses accurately reflect what would be observed in the real
environment. Methods used in validation can include comparing simulated outputs
to empirical data, using statistical tests to check for significant deviations,
and conducting sensitivity analyses to see if the model behaves consistently
when variables are adjusted. Validation provides confidence that the simulation
model is a reliable tool for making predictions and deriving insights about the
real system.

Validation of the model will be done using the experiments described in
\chref{ch:pocs}, by checking that the results of the simulations are consistent
with theoretical and empirical data drawn from the real Bitcoin network and
from other related works. This chapter focuses on the verification tests
conducted on \iblock{}.

The configurations for the verification tests presented in this
chapter are available in the \texttt{simulations/tests.ini} file of the
repository.

\section{Manual Verification}\label{sec:manual-verification}

The \texttt{SimpleTest} configuration was used to conduct detailed manual
checks on the model. This configuration features a small network of three
nodes, including one miner node. Each node generates transactions at a rate of
one transaction per minute, providing a simplified environment for focused
inspection.

A graphical test was performed within the \omnetpp{}'s QT environment, enabling
a visual review of network behavior. This test allowed for real-time
observation of interactions, confirming that nodes communicated and processed
messages as expected. Additionally, the model was run step-by-step under a
debugger to closely examine the logic and data flow, ensuring each component
operated as intended and followed the correct sequence of operations.

The \omnetpp's event trace log was generated and analyzed to verify the
scheduling of events, confirming that they occurred in the expected order and
timeline. This careful, multi-faceted approach provided a thorough assessment
of the model's core workflow, confirming that fundamental functions and data
pathways were correctly implemented.

\section{Smoke Tests}\label{sec:smoke-tests}

Smoke tests are preliminary tests that ensure the software launches and runs
for extended periods without crashing. Throughout the model's code, exceptions
are programmed to trigger when the model deviates from expected states, causing
a crash and subsequent smoke test failure. Additionally, \omnetpp{} may throw
exceptions if the model is improperly implemented.

This test utilized a network of 10 nodes, including three miners. All nodes,
both miners and non-miners, generated transactions at a rate of one every five
seconds per node. The model was run for a simulated period of 20 days, allowing
for an evaluation of the network's difficulty adjustment occurring every 2016
blocks.

To conserve computational resources, statistics collection was disabled during
this test. Statistics collection will be addressed in \chref{ch:pocs},
demonstrating how \iblock{} can be used to execute simulations and gather
statistical data.

The test concluded successfully without any errors. Note also that every test
or simulation conducted, including those detailed in this chapter and in
\chref{ch:pocs}, can also be considered a smoke test.

\section{Memory Leak Detection}\label{sec:memory-tests}

The system was checked for memory leaks using the Memcheck tool \cite{memcheck}
of the Valgrind suite \cite{valgrind}. A larger network of 50 nodes, 10 of
which were miners, was tested, with all nodes generating transactions at a rate
of one every 15 seconds. The model was run for one hour of simulated time.
Valgrind significantly slows down execution (especially with debug mode
enabled), limiting the run length, though one hour should be sufficient to
detect memory leaks.

Notably, \omnetpp{} also performs internal checks for potential memory leaks in
any objects extending the \texttt{cOwnedObject} class, extensively used in the
system's implementation.

Both Valgrind and \omnetpp{} reported no memory leaks. Valgrind's output,
displayed below, indicates only that some memory allocated by the standard C++
library and \omnetpp{} remains unreleased. This memory is \textit{still
reachable} and will be freed when the program exits. Also it is totaling only
about 75 Kbytes --- an insignificant amount compared to the total memory usage
of the software.

\begin{verbatim}
==20550== LEAK SUMMARY:
==20550==    definitely lost: 0 bytes in 0 blocks
==20550==    indirectly lost: 0 bytes in 0 blocks
==20550==      possibly lost: 0 bytes in 0 blocks
==20550==    still reachable: 75,218 bytes in 883 blocks
==20550==         suppressed: 0 bytes in 0 blocks
\end{verbatim}

\section{Testing Determinism with OMNeT++
Fingerprints}\label{sec:deterministic-tests}

Deterministic tests involve multiple runs with identical configurations and
random number generator seeds, verifying that each simulation run yields the
\textit{exact} same results, a critical requirement for reproducibility in
simulations.

For this test, a configuration of 10 nodes was used, including three miners,
with each node generating an average of one transaction every 10 seconds.

This test leverages an \omnetpp{} feature that computes a simulation
``fingerprint'', a hash updated throughout the simulation based on various
event properties, and displayed at the simulation's conclusion. \omnetpp{} then
compares the final fingerprint with a predefined expected value specified in
the configuration file. Ingredients used in the fingerprint calculation can be
set in the configuration file \cite[Section~15.4]{omnetpp-simulation-manual}.

All statistics collection was enabled, and each run simulated 24 hours.

In the initial run to generate the fingerprint, the following components were
included for calculation:
\begin{multicols}{2}
	\begin{itemize}
		\item The event number;
		\item The simulation time;
		\item The message/event full name;
		\item The message (packet) bit length;
		\item The module full path;
		\item The random numbers drawn;
		\item The scalar results;
		\item The statistic results (histograms, etc.);
		\item The vector results.
	\end{itemize}
\end{multicols}

These components were selected to capture key properties of each simulation
aspect and object. The last three items --- scalar results, statistical results,
and vector results --- are especially critical for ensuring reproducibility.
Anyway, also other ingredients were included in order to guarantee the
simulation's trajectory is always \textit{exactly} the same.

The generated fingerprint was:
\begin{verbatim}
5250-e519/etnlprszv
\end{verbatim}

With this fingerprint set in the configuration, the simulation was repeated 100
times. All 100 runs successfully matched the expected fingerprint, with each
output file containing the ``Fingerprint successfully verified'' message:
\begin{verbatim}
$ grep -c 'Fingerprint successfully verified' *.out | wc -l
100
\end{verbatim}

