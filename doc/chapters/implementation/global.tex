\section{Global Modules}\label{sec:impl-global}

In \figref{fig:app-uml} on page \pageref{fig:app-uml}, the green-highlighted
classes represent the global modules within the simulation. These modules,
external to individual nodes, provide essential services across the network.

Global modules identify nodes and applications in the network through
\omnetpp{}'s topology discovery and NED properties. For example, each network
node has a \code{@bitcoinNode} property, each wallet is tagged with
\code{@wallet}, and each \code{BlockchainManager} is marked with
\code{@blockchainManager} in their NED definitions.

The \code{NodeManager} module serves as a directory of all nodes, maintaining
an array of them accessible via the \code{nodes()} method, which allows for
node iteration in loops. This module is used by \code{BlockchainManager} and
\code{MempoolManager} for block and transaction propagation across the network.

The \code{WalletManager} module supports the \code{TransactionGenerator}
application by supplying new Bitcoin addresses (instances of the
\code{BitcoinAddress} class) for random wallets throughout the network.

The \code{GBM} module is slightly more complex and it is described in the
following section.

\subsection{The Global Blockchain Manager}\label{subsec:impl-global-gbm}

The \code{GBM} (Global Blockchain Manager) module is responsible for creating
the \emph{genesis block} and managing memory by cleaning out obsolete blocks.

In a Bitcoin network, the genesis block is the initial block in the blockchain.
Unlike regular blocks, it lacks standard transactions since there's no
pre-existing Bitcoin in circulation; only the coinbase transaction generates
the first coins. In the Bitcoin protocol, this coinbase transaction typically
contains 50 bitcoins, but in \iblock{}, the genesis block's coin amount can be
adjusted. Additionally, these initial funds can be distributed across multiple
wallets in the network based on user-defined starting balances.

When setting up the simulation, the user specifies each wallet's initial
balance in the simulation configuration. The \code{GBM} then assigns these
amounts within a single coinbase transaction in the genesis block, with each
output directed to a designated wallet. The \code{GBM} then distributes the
genesis block to all \code{BlockchainManager} instances in the network using
the \code{addGenesisBlock()} method. In this way, when the simulation starts,
each wallet will have its initial balance given by the genesis block.

Once the genesis block is established in all network blockchains, the
simulation begins. The \code{GBM} module periodically wakes to perform memory
cleanup, freeing space from blocks (and transactions) that are no longer
necessary.

The cleanup process is as follows:
\begin{enumerate}
	\item Each time a \code{BlockchainManager} adds a new block to its main
		blockchain branch, it notifies the \code{GBM}. The \code{GBM}
		stores a reference to this block in a map tracking all main
		branches across the network;
	\item On each \code{GBM} wake (configurable interval), it identifies
		the common ancestor of all main branches. The \code{GBM} then
		retains this common ancestor and its parent, discarding all
		older blocks. Deletion is managed by removing the parent
		pointer from the earliest retained block. Thanks to smart
		pointers, all blocks without remaining references are
		automatically deleted;
	\item The \code{GBM} then calls each \code{BlockchainManager}'s
		\code{cleanup()} method, passing the height of the retained
		common ancestor. Each \code{BlockchainManager} then removes any
		branches whose head height is below this threshold.
\end{enumerate}

By using smart pointers, blocks are deleted simply by removing references, and
transactions within blocks are also automatically removed once they are no
longer referenced by any block or any mempool.

The \code{GBM} module has a single configurable parameter:
\code{cleanupInterval}, which defines the time between memory cleanup
operations. By default, this interval is set to one hour (simulated time).

This memory management approach allows \iblock{} to handle large-scale
simulations efficiently, maintaining essential blocks and transactions without
memory overflow. If an applications must ensure that a block or a transaction
is retained in memory, it just need to keep a reference to it.
