\chapter{Conclusions}\label{ch:conclusions}

In this work, we proposed \iblock{}, a model library for \omnetpp{} designed to
simulate complex and large-scale Bitcoin networks with complete internal
representation of blocks and transactions. The blockchain data structures in
\iblock{} closely mirror the actual Bitcoin data structures, enabling
high-fidelity network simulations. This level of detail is unique among
existing Bitcoin simulators, as none currently support the UTXO model, which
\iblock{} implements in the \code{Wallet} application \cite{simureview}.
Furthermore, \iblock{} offers numerous features that set it apart, possibly
making it a valuable tool for researchers and developers working on Bitcoin or
other blockchain-based applications.

A key advantage of \iblock{} lies in its foundation on \omnetpp{}. Unlike many
Bitcoin simulators that lack a structured framework, extending \iblock{} is
fast and straightforward. For example, we demonstrated how easily new
behaviors, such as the selfish mining strategy, can be implemented by extending
the base classes of \iblock{} and using the tools provided by \omnetpp{}.
Leveraging \omnetpp{} also allows \iblock{} to integrate seamlessly with other
libraries and models developed for the same framework, such as INET, which
facilitates the simulation of complex scenarios and allows to study the
behavior of the network from different perspectives.

Performance-wise, \iblock{} benefits significantly from being developed in C++,
which compiles to native code. This provides a considerable advantage over
other simulators often built with higher-level, (semi-)interpreted languages
such as Python, Java, Go, or Scala. As a result, \iblock{} is generally faster
and more memory-efficient. When compared with BlockSim, one of the most widely
used Bitcoin simulators, \iblock{} demonstrated superior performance against
BlockSim's ``Full'' mode --- processing faster and consuming less memory ---
while also providing a higher level of simulation detail and collecting more
statistics.

Despite its strengths, \iblock{} is not yet a fully complete Bitcoin simulator.
It currently lacks support for \omnetpp{} links and channels, as well for
certain aspects of the Bitcoin protocol. However, these limitations do not
diminish its valua as a foundation for further development, and still it can be
used to simulate creation and propagation of blocks and transactions in a
Bitcoin network. We believe that \iblock{} also represents a strong starting
point for building a comprehensive Bitcoin simulator capable of supporting
every aspect of the Bitcoin protocol in detail.
