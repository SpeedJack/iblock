\section{Integration with INET}\label{sec:INET}

One of the significant advantages of using the \omnetpp{} framework is its
flexibility to integrate diverse simulation models. This enables the creation
of complex scenarios that analyze system behavior across multiple levels of
abstraction.

The \textbf{INET Framework} is an open-source model library for the \omnetpp{}
simulation environment. It provides protocols, agents, and other models for
researchers and students working with communication networks. INET contains
models for the Internet stack (TCP, UDP, IPv4, IPv6, OSPF, BGP, etc.), wired
and wireless link-layer protocols (Ethernet, PPP, IEEE 802.11, etc.), support
for mobility, MANET protocols, DiffServ, MPLS with LDP and RSVP-TE signaling,
several application models, and many other protocols and components
\cite{inet}.

By integrating \iblock{} with INET, users could simulate the full TCP/IP stack
while nodes exchange Bitcoin blocks and transactions. This would provide deeper
insights into the interaction between Bitcoin's protocol and underlying network
dynamics, such as congestion, packet loss, and latency, which are critical for
studying real-world scenarios.

\iblock{} has been designed with future INET integration in mind. In fact, the
name \iblock{} itself is derived from ``INET'', emphasizing this planned
synergy. Once the implementation of Bitcoin's full protocol, including support
for \omnetpp{} links, is completed, INET is expected to work seamlessly with
\iblock{}.

The \code{MessageDispatcher} module developed in \secref{sec:partial-work} is
central to this integration. This module is designed to be adaptable, capable
of handling both \omnetpp{} links and INET-based communication. As such, it
forms the foundation for using INET's advanced networking models alongside
\iblock{}'s blockchain simulations.

Combining \iblock{} and INET would unlock the ability to:
\begin{enumerate}
	\item Simulate Bitcoin transactions and block propagation with
		detailed, realistic network environments;
	\item Study the effects of bandwidth limitations, latency, and other
		network parameters on blockchain performance;
	\item Analyze packets as they traverse the TCP/IP stack, from the
		application layer down to the physical layer;
	\item Explore complex scenarios such as congestion attacks or the
		impact of specific routing protocols on the Bitcoin network.
\end{enumerate}

This integration would significantly enhance the applicability of \iblock{},
making it an invaluable tool for researchers examining the interplay between
blockchain protocols and network infrastructures.
