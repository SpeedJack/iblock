\section{Memory Leak Detection}\label{sec:memory-tests}

The system was checked for memory leaks using the Memcheck tool \cite{memcheck}
of the Valgrind suite \cite{valgrind}. A larger network of 50 nodes, 10 of
which were miners, was tested, with all nodes generating transactions at a rate
of one every 15 seconds. The model was run for one hour of simulated time.
Valgrind significantly slows down execution (especially with debug mode
enabled), limiting the run length, though one hour should be sufficient to
detect memory leaks.

Notably, \omnetpp{} also performs internal checks for potential memory leaks in
any objects extending the \texttt{cOwnedObject} class, extensively used in the
system's implementation.

Both Valgrind and \omnetpp{} reported no memory leaks. Valgrind's output,
displayed below, indicates only that some memory allocated by the standard C++
library and \omnetpp{} remains unreleased. This memory is \textit{still
reachable} and will be freed when the program exits. Also it is totaling only
about 75 Kbytes --- an insignificant amount compared to the total memory usage
of the software.

\begin{verbatim}
==20550== LEAK SUMMARY:
==20550==    definitely lost: 0 bytes in 0 blocks
==20550==    indirectly lost: 0 bytes in 0 blocks
==20550==      possibly lost: 0 bytes in 0 blocks
==20550==    still reachable: 75,218 bytes in 883 blocks
==20550==         suppressed: 0 bytes in 0 blocks
\end{verbatim}
