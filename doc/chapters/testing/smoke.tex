\section{Smoke Tests}\label{sec:smoke-tests}

Smoke tests are preliminary tests that ensure the software launches and runs
for extended periods without crashing. Throughout the model's code, exceptions
are programmed to trigger when the model deviates from expected states, causing
a crash and subsequent smoke test failure. Additionally, \omnetpp{} may throw
exceptions if the model is improperly implemented.

This test utilized a network of 10 nodes, including three miners. All nodes,
both miners and non-miners, generated transactions at a rate of one every five
seconds per node. The model was run for a simulated period of 20 days, allowing
for an evaluation of the network's difficulty adjustment occurring every 2016
blocks.

To conserve computational resources, statistics collection was disabled during
this test. Statistics collection will be addressed in \chref{ch:pocs},
demonstrating how \iblock{} can be used to execute simulations and gather
statistical data.

The test concluded successfully without any errors. Note also that every test
or simulation conducted, including those detailed in this chapter and in
\chref{ch:pocs}, can also be considered a smoke test.
