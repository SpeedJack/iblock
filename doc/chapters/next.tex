\chapter{Future Work}\label{ch:next}

While \iblock{} successfully simulates the construction and propagation of
blocks and transactions in a Bitcoin network, the Bitcoin protocol encompasses
far more than these basic operations. Expanding \iblock{} to incorporate
additional protocol features would enhance its utility, enabling a broader
range of experiments and deeper insights into Bitcoin and blockchain dynamics.

The most critical feature currently absent in \iblock{} is the implementation
of Bitcoin's \textbf{gossip protocol}. In a real Bitcoin network, each node
connects to a limited number of peers. When a node receives a block or
transaction, it notifies its peers via \emph{inventory messages}. These peers
can then decide whether to request the data using a \code{GETDATA} message,
depending on whether they already possess it or not. This propagation process
continues until the entire network is updated. Incorporating this mechanism
would provide a more accurate simulation of Bitcoin's peer-to-peer
communication and its potential bottlenecks or inefficiencies. Furthermore,
implementing the gossip protocol would enable the study of other types of
attacks, such as the \emph{eclipse attack}, where an attacker isolates a node
by controlling its connections and feeds it with false informations to perform,
for example, a double-spending attack against the victim \cite{eclipse}.

Beyond the gossip protocol, other key Bitcoin mechanisms remain to be
implemented, including:
\begin{description}
	\item[Peer discovery] Nodes query peers for lists of other nodes to
		expand their connections;
	\item[Bloom filters] Utilized by Simplified Payment Verification (SPV)
		nodes to query the blockchain for specific transactions while
		maintaining user privacy;
	\item[Segregated Witness (SegWit)] A protocol improvement that
		separates transaction signatures from the transaction data to
		optimize storage and reduce block sizes (and thus fees);
	\item[Other Protocol Features] Elements like transaction malleability
		prevention, multi-signature schemes, and advanced scripting
		capabilities are essential for comprehensive protocol support.
\end{description}
A complete Bitcoin simulator would integrate these features, providing the
capability to study nuanced protocol behaviors and optimizations.

The next logical step in \iblock{}'s evolution is to implement support for
\omnetpp{}'s links. This addition would enable the simulation of a more
realistic network environment, where nodes communicate over channels with
varying bandwidths and latencies. Such functionality is critical for modeling
real-world network conditions and is a prerequisite for accurately simulating
the gossip protocol. Although partial progress has been made in this area, as
discussed in \secref{sec:partial-work}, completing this feature remains a
priority.

Looking forward, \iblock{} could be extended to simulate other blockchain
systems, such as Ethereum \cite{ethereum}. Ethereum introduces unique concepts
like Proof-of-Stake, smart contracts, gas mechanisms, and an account-based
model, which differ significantly from Bitcoin's UTXO-based approach.
Developing a more generic framework capable of supporting multiple blockchain
systems would be a significant milestone. Such a framework could allow
researchers and developers to simulate diverse blockchain ecosystems under a
unified architecture, broadening \iblock{}'s applicability.

By incorporating these enhancements, \iblock{} can evolve into a robust,
versatile simulator for blockchain research. Features such as the gossip
protocol, peer discovery, and SegWit, combined with support for \omnetpp{}
links and channels, will deepen the simulation's fidelity. Furthermore,
extending the framework to support systems like Ethereum would position
\iblock{} as a universal tool for studying blockchain-based networks, unlocking
new opportunities for research and development.

\section{Integration with INET}\label{sec:INET}

One of the significant advantages of using the \omnetpp{} framework is its
flexibility to integrate diverse simulation models. This enables the creation
of complex scenarios that analyze system behavior across multiple levels of
abstraction.

The \textbf{INET Framework} is an open-source model library for the \omnetpp{}
simulation environment. It provides protocols, agents, and other models for
researchers and students working with communication networks. INET contains
models for the Internet stack (TCP, UDP, IPv4, IPv6, OSPF, BGP, etc.), wired
and wireless link-layer protocols (Ethernet, PPP, IEEE 802.11, etc.), support
for mobility, MANET protocols, DiffServ, MPLS with LDP and RSVP-TE signaling,
several application models, and many other protocols and components
\cite{inet}.

By integrating \iblock{} with INET, users could simulate the full TCP/IP stack
while nodes exchange Bitcoin blocks and transactions. This would provide deeper
insights into the interaction between Bitcoin's protocol and underlying network
dynamics, such as congestion, packet loss, and latency, which are critical for
studying real-world scenarios.

\iblock{} has been designed with future INET integration in mind. In fact, the
name \iblock{} itself is derived from ``INET'', emphasizing this planned
synergy. Once the implementation of Bitcoin's full protocol, including support
for \omnetpp{} links, is completed, INET is expected to work seamlessly with
\iblock{}.

The \code{MessageDispatcher} module developed in \secref{sec:partial-work} is
central to this integration. This module is designed to be adaptable, capable
of handling both \omnetpp{} links and INET-based communication. As such, it
forms the foundation for using INET's advanced networking models alongside
\iblock{}'s blockchain simulations.

Combining \iblock{} and INET would unlock the ability to:
\begin{enumerate}
	\item Simulate Bitcoin transactions and block propagation with
		detailed, realistic network environments;
	\item Study the effects of bandwidth limitations, latency, and other
		network parameters on blockchain performance;
	\item Analyze packets as they traverse the TCP/IP stack, from the
		application layer down to the physical layer;
	\item Explore complex scenarios such as congestion attacks or the
		impact of specific routing protocols on the Bitcoin network.
\end{enumerate}

This integration would significantly enhance the applicability of \iblock{},
making it an invaluable tool for researchers examining the interplay between
blockchain protocols and network infrastructures.

